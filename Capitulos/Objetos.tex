%---------------------------------------------------------------------
%
%                          Cap�tulo 10
%
%---------------------------------------------------------------------

\chapter{Dise�o e Implementaci�n de los objetos}

\begin{FraseCelebre}
\begin{Frase}
	Un Anillo para gobernarlos a todos.
	Un Anillo para encontrarlos, un Anillo para atraerlos a todos y atarlos en las tinieblas.
\end{Frase}
\begin{Fuente}
	Tolkien, 1993
\end{Fuente}
\end{FraseCelebre}

Objetos... cuan de importantes son en muchas historias. Quien no recuerda a Exc�libur, la legendaria espada del rey Arturo, o al Anillo �nico de Sauron en los libros de Tolkien. Los objetos pueden ser el objetivo de una historia o dar ese giro inesperado que cambia el transcurso de la historia. En este proyecto hemos querido introducir objetos para crear historia mucho m�s ricas.
\label{cap:Objetos}

\section{Dise�o de los objetos}
En este punto del proyecto la historias que se generan tienen peso y dinamismo por todo lo que se ha explicado en puntos anteriores, pero con la idea de seguir consiguiendo m�s enriquecimiento de historias se ha decidido introducir objetos. Esto se introduce aparte de para conseguir historias m�s complejas tambi�n reflejar mejor algunos conflictos entre personajes, ya que si en un enfrentamiento entre dos personajes uno anteriormente ha conseguido una espada, este tendr� m�s oportunidades de salir vencedor.

\subsection{Objetos consumibles}
El primer planteamiento que se tuvo en mente de los objetos fue el de que fuesen objetos para usar en el momento en el que se encuentran obligatoriamente. Estos objetos encontrados se consumir�n en el acto y desaparecer�n del mapa, alterando las estad�sticas del personaje en el momento de su consumici�n (introducir la refencia de stats de personajes).

Este planteamiento sirvi� como primer paso para introducir los objetos en el proyecto, aunque se pens� que los objetos dise�ados de esta manera quedaban en la historia de forma ef�mera.

\subsection{Objetos clave}
Con este pensamiento se introdujo un segundo tipo de objetos que ser�an clave durante el transcurso de la historia para poder lograr los objetivos de algunos personajes. Por ejemplo para que el caballero pueda entrar en la monta�a para rescatar a la princesa necesitar� un objeto que le servir� como llave para poder entrar.
Gracias al dise�o de este nuevo tipo de objetos, habr� objetos que tendr�n un papel relevante en la historia, al igual que los objetos que se han mencionado anteriormente en la introducci�n de este capitulo.

\subsection{Objetos creados por los PNJ's}
\label{cap:Objetos:ObjetosPNJ}
En este punto ya se tienen dise�ados dos tipos de objetos que cubren las necesidades actuales del proyecto y se observa que tanto los personajes de tipo protagonista como los de tipo antagonista interact�an con los objetos tal y como est�n dise�ados hasta este punto, pero los PNJ. Debido a esto y ya que los PNJ est�n configurados como entidades que tienen un oficio, no es descabellado hacer que estos fabriquen objetos, como que un alquimista cree una poci�n, un tabernero sirva una cerveza o un armero fabrique un arma. Estos objetos fabricados por los PNJ est�n pensados para tener un funcionamiento id�ntico a los \textbf{objetos consumibles} es por esto que los objetos que fabriquen los PNJ solo se aparecer�n en la historia a trav�s de dicho PNJ, es decir no se podr� encontrar mientras se explora por el mapa, y de esta forma diferenciar ambos tipos de objetos.

\section{Implementaci�n de los objetos}
Como el resto de componentes de la aplicaci�n, los objetos han sido implementados de forma altamente configurable y extensible. Se va a explicar detalladamente los pasos seguidos para conseguirlo.
 
\subsection{Sistema de clases}

\subsection{Carga en el sistema}

\subsection{Introducci�n en la historia}

\subsection{Funcionalidades}
