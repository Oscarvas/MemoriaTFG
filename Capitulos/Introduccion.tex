%---------------------------------------------------------------------
%
%                          Cap�tulo 1
%
%---------------------------------------------------------------------

\chapter{Introducci�n}

\begin{FraseCelebre}
\begin{Frase}
\end{Frase}
\begin{Fuente}
\end{Fuente}
\end{FraseCelebre}

El problema expuesto por nuestros compa�eros sobre la dificultad de generar una historia y que adem�s esta sea interesante nos afecta de la misma manera que a ellos lo que nos llev� a hacernos algunas preguntas:

�C�mo podemos simular una caracter�stica como la creatividad?
%eliminar la pregunta
Para narrar una historia entretenida se requiere de una habilidad para hacerla �nica, la creatividad. 
La tarea de recrear habilidades humanas artificialmente es una tarea compleja. Por tanto nuestro cometido ha sido asaltar la creatividad computacional para poder generar historias.

�Qu� hace que una historia sea interesante?
%eliminar la pregunta
A lo largo de nuestra vida hemos visto como novelas han tenido un gran �xito, como puede ser ``El se�or de los anillos'' o ``Canci�n de hielo y fuego'' entre otras, y al pensar en estas novelas observamos que tienen algunos factores en com�n. 
Uno de estos factores es que no exista un �nico protagonista, si no que todos lo personajes importantes en la historia tienen un objetivo y depende del escritor si los personajes colaboran o entran en conflicto para conseguir el objetivo. Para poder obtener el plan a la hora de la consecuci�n de estos objetivos utilizaremos un planificador, que nos devolver� el plan que tenemos que seguir para conseguir nuestro objetivo en funci�n de el estado en el que se encuentre el personaje.
Tambi�n es importante que no todos los personajes de la historia tengan un objetivo, ya que se podr�a perder el hilo argumental, si no que aparecen en la historia para interactuar con alguno de los protagonistas en momentos determinados. %no me gusta esta parte que viene a continuacion, la de los videojuegos.
 Estos personajes los denominamos con un t�rmino del mundo de los videojuegos, PNJ(Peronaje No Jugable).
Otro factor com�n que observamos es el mapa donde se desarrollan y es que no son entornos muy reducidos.

�C�mo hacer con un mismo entorno historias diversas?
%eliminar la pregunta
Al pensar en la diversidad de historias, nos viene a la mente los juegos de rol. Un juego de rol es aquel en donde el jugador desempe�a el rol de un personaje ficticio y existe un director que introduce a los personajes dentro de la historia. A pesar de estar en un mismo entorno observamos historias diferentes cada vez que se juega. La diferencia de cada personaje en los juegos de rol viene marcada por los rasgos antropol�gicos, atributos f�sicos, habilidades determinadas, etc�tera. Todos estos rasgos de un personaje como el entorno donde se introduce(mapa, PNJ, objetos) y la narrativa al interactuar con �l es configurable mediante archivos.

Con estas preguntas en mente y buscando sus respuestas hemos desarrollado un generador de historias diversas mediante simulaciones, introduciendo un estado inicial en archivos de configuraci�n.

%por lo gnral  no me gusta demasiado la introduccion, creo que resume mas nuestro TFG que introducirlo, quiz� no sea el objetivo, opinad vosotros tambien.
