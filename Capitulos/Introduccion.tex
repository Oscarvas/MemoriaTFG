%---------------------------------------------------------------------
%
%                          Cap�tulo 1
%
%---------------------------------------------------------------------

\chapter{Introducci�n}

\begin{FraseCelebre}
\begin{Frase}
\end{Frase}
\begin{Fuente}
\end{Fuente}
\end{FraseCelebre}

El problema sobre la dificultad de generar una historia y que adem�s esta sea interesante afecta directamente al proyecto, por lo que en primer lugar hab�a que responder algunas preguntas:\\
\\
\textbf{\textit{�C�mo recrear una cualidad como la creatividad?}}\\
%eliminar la pregunta
Para narrar una historia entretenida se requiere de una habilidad para hacerla �nica, la creatividad. 
La tarea de recrear habilidades humanas artificialmente es un problema complejo y m�s a�n cuando estas habilidades son art�sticas. Por tanto el cometido ha sido asaltar la creatividad computacional para poder generar historias.\\
\\
\textbf{\textit{�Qu� hace que una historia sea interesante?}}\\
%eliminar la pregunta
A lo largo de nuestra vida hemos visto como novelas han tenido un gran �xito, como puede ser ``El se�or de los anillos'' o ``Canci�n de hielo y fuego'' entre otras, y al pensar en estas novelas observamos que tienen algunos factores en com�n. 
Uno de estos factores es que no exista un �nico protagonista, si no que la trama gira en torno a varios personajes de la historia. Estos tienen un objetivo marcado y depende del escritor si los personajes colaboran o entran en conflicto para lograr la consecuci�n del objetivo. Para poder saber los requisitos que se necetian para la consecuci�n de estos objetivos se utiliza un planificador, el cual realiz� un plan, en funci�n del estado en el que se encuentre el personaje, que se tendr� que seguir para alcanzar el objetivo .
Tambi�n es importante que no todos los personajes de la historia tengan un objetivo, ya que se podr�a perder el hilo argumental, si no que aparezcan en la historia para interactuar con alguno de los protagonistas en momentos determinados. %no me gusta esta parte que viene a continuacion, la de los videojuegos.
 Estos personajes los denominamos con un t�rmino del mundo de los videojuegos, PNJ(Peronaje No Jugable).
Otro factor com�n que observamos es el mapa donde se desarrollan y es que no son entornos muy reducidos.\\
\\
\textbf{\textit{�C�mo hacer historias diversas dentro un mismo entorno?}}\\
%eliminar la pregunta
Al pensar en la diversidad de historias, nos viene a la mente los juegos de rol. Un juego de rol es aquel en donde el jugador desempe�a el rol de un personaje ficticio y existe un director que introduce a los personajes dentro de la historia. A pesar de estar en un mismo entorno observamos historias diferentes cada vez que se juega. La diferencia de cada personaje en los juegos de rol viene marcada por los rasgos antropol�gicos, atributos f�sicos, habilidades determinadas, etc�tera. Todos estos rasgos de un personaje como el entorno donde se introduce(mapa, PNJ, objetos) y la narrativa al interactuar con �l es configurable mediante archivos.

Con estas preguntas en mente y buscando sus respuestas hemos desarrollado un generador de historias diversas mediante simulaciones, introduciendo un estado inicial en archivos de configuraci�n.

%por lo gnral  no me gusta demasiado la introduccion, creo que resume mas nuestro TFG que introducirlo, quiz� no sea el objetivo, opinad vosotros tambien.
