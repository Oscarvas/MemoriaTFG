%---------------------------------------------------------------------
%
%                          Cap�tulo 1
%
%---------------------------------------------------------------------

\chapter{Introducci�n}

\begin{FraseCelebre}
\begin{Frase}
...
\end{Frase}
\begin{Fuente}
...
\end{Fuente}
\end{FraseCelebre}

El problema expuesto por nuestros compa�eros sobre la difucultad de generar una historia, que adem�s esta historia sea interesante nos afecta de la misma manera que a ellos pero con un a�adido que nosotros ademas queremos que la probabilidad de repetici�n de una historia sea infima, con lo que nos planteamos una pregunta: �c�mo conseguir diversidad de historias interesantes y que adem�s sean aleatorias?

Esta pregunta nos hace pensar en las novelas que hemos le�do como puede ser 'El se�or de los anillos' o 'Canci�n de hielo y fuego' entre otras y lo que las hace interesante no es el protagonista como tal, ya que no exite un �nico protagonista, si no que todos lo personajes importantes en la historia tienen un objetivo y que adem�s si este objetivo entra ne conflicto con el de los otros personajes la historia toma mayor interes. Para poder obtener el plan a la hora de la consecuci�n de estos objetivos utilizaremos un planificador, que nos devolver� el plan que tenemos que seguir para conseguir nuestro objetivo en funci�n de el estado en el que se encuentr� el personaje.

Con uno de los objetivos resueltos, nos toca pensar en el factor de aleatoriedad y en nuestro entorno donde podemos ver historias aleatorias pero siempre dentro del mismo universo ficticio lo que nos lleva a los juegos de rol.
Los juegos de rol son aquellos en donde el jugador desempe�a el rol de un personaje ficticio y es el director o master en que introduce a este personaje dentro de la historia que se est� narrando. La diferencia de cada personaje viene marcada por los rasgos antropol�gicos, atributos f�sicos, habilidades determinadas, etc�tera.
Pero esto nos lleva a un nuevo problema y es que si siempre se crean los personajes, estan en un camino lineal en el que el objetivo es de ir de un punto a otro y volver la historia siempre ser� la misma, por lo que est� directamente relacionado el tama�o del mundo de la historia con la cantidad de personajes incluidos, adem�s de que si el personaje creado es el m�s fuerte siempre ser� el m�s fuerte.
Con este nuevo problema sobre la mesa tenemos la idea que mucha veces la epicidad de los heroes en las historias en m�s por un objeto que por el propio personaje, con lo que la inclusi�n de objetos ser� importante para dar mayor n�mero de historias. Por lo tanto todos estos factores mencionados son los que utilizaremos para que afecten a la simulaci�n de nuestra historia. 
