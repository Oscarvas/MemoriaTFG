%---------------------------------------------------------------------
%
%                          Cap�tulo 3
%
%---------------------------------------------------------------------
\chapter{Objetivos}


\begin{FraseCelebre}
\begin{Frase}
	Algunas veces hay que decidirse entre una cosa a la que se est� acostumbrado y otra que nos gustar�a conocer
\end{Frase}
\begin{Fuente}
	Paulo Coehlo
\end{Fuente}
\end{FraseCelebre}

\label{cap:3:Objetivos}

%hay que cambiar esta introduccion
Al disponer de un sistema base capaz de generar historias sencillas, es necesario realizar un estudio completo del sistema. El prop�sito de este estudio es determinar con exactitud las herramientas y posibilidades que nos ofrece el sistema, y as� poder planificar las mejoras que hiciesen falta realizar sobre el mismo.%quiero escribir algo mas aqui


%-------------------------------------------------------------------
\section{Motivaciones}
%-------------------------------------------------------------------

A medida que se van haciendo diferentes pruebas sobre el sistema del que partimos, varias carencias y posibilidades de mejora se hacen presentes:
\begin{itemize}
	\item Nos encontramos frente a un sistema altamente cerrado. La posibilidad de ampliar o modificar la funcionalidad del sistema se encuentra fuertemente ligada al c�digo, y no se dispone de una forma sencilla para introducir peque�os cambios sin tener que modificar el c�digo. Si queremos aumentar la escala del sistema, es primordial reducir dr�sticamente la cantidad de ataduras que existen con el c�digo.
	
	\item Carencia de variedad en la generaci�n de historias. La sencillez del sistema afecta que no exista mucha diversidad en las posibles historias que se generan. 
	
	\item �Siempre tendr�amos que ver el mismo entorno de historia?. En relaci�n a lo comentado anteriormente, no es posible variar el entorno de la historia, con lo cual estamos obligados a ver que todas las historias ocurren en el mismo entorno (medieval).
	
	\item �Generar m�s contenido sobre lo mismo? � �Generar diferentes tipos de historias sencillas?. Son algunas de las primeras cuestiones planteadas y sobre las cuales hubo un largo per�odo de discusi�n. 
	
\end{itemize}


Resumiendo, se quiere crear un sistema que sea capaz de generar historias. Esta aplicaci�n debe estar basada en el paradigma de agentes y a su vez usar un sistema de planificaci�n que permita a los agentes interactuar y crear la historia. Tenemos un sistema que usa todo lo anterior, pero es un sistema muy b�sico que principalmente genera la misma historia siempre, con peque�as modificaciones. Ahora necesitamos un sistema que permita al usuario poder generar historias din�micas, que disponga de la oportunidad de poder cambiar el entorno de la historia, que el usuario sea capaz de incluir nuevos elementos durante la trama, que �ste tenga la posibilidad de crear nuevos personajes y alterar la cantidad de personajes que van a interactuar en cada historia generada. Un sistema que se adapte a la creatividad de cada usuario.

%-------------------------------------------------------------------
\section{Objetivos}
%-------------------------------------------------------------------

Nuestro principal objetivo es el de desarrollar una aplicaci�n capaz de generar historias centradas en los personajes mediante simulaciones. Estas simulaciones de historias queremos que tengan un factor de repetici�n bajo y as� conseguir que cada historia sea totalmente distintiva, ``�nica'' Adem�s para poder tener mayor complicidad con el usuario/lector, tendr� la opci�n de crear y configurar ciertas parte de la historia, como los personajes que act�an en ella, el nombre de los lugares donde esta transcurre, que tipos de monstruos quiere que aparezcan en su mundo... Pero ya entraremos en eso mas adelante.

Todos los personajes que se utilizan en las simulaciones son agentes, que trataremos a trav�s de un Sistema Multiagentes. Cada personaje tiene unos  objetivos que intentar cumplir y la narraci�n de todo el trascurso de acciones desde su creaci�n hasta alcanzar su objetivo o morir en el intento nos proporcionar� fragmentos de la historia generada. Para poder saber cual es el mejor camino para alcanzar los objetivos y evitar que nuestra historia pierda inter�s utilizaremos un planificador.

Al tener un trabajo previo, es posible que haya reutilizaci�n de c�digo,  en el apartado de reutilizaci�n. Aunque hay algo que si queremos diferenciar desde el principio y es que ellos en los objetivos dicen "que al aumentar el trabajo se observar�an variaciones en la historia, aunque el fin fuese el mismo", y nosotros queremos evitar eso.

Por ello los objetivos que nos marcamos son:
\begin{itemize}
	\item Dise�ar un generador de historias, en el cual cada historia sea "�nica" y no tenga un final predefinido. Para ello:
	\begin{itemize}
		\item Ampliaremos el mapa.
		\item Todos los personajes principales podr�n realizar acciones.		
		\item Introduciremos objetos.
		\item Introduciremos PNJ.
	\end{itemize}
	\item Dise�ar una interfaz interactiva con la que queremos sumergir al usuario/lector en la historia.
	\item Introducir replanificaciones en la historia cuando ocurran eventos significativos.
	\item Utilizar archivos de configuraci�n para favorecer a la aleatoriedad de las historias, guardando el mapa, los objetos, los PNJ's.
	\item Utilizar archivos de configuraci�n que nos permita cambiar el �mbito de las historias generadas.
\end{itemize}

