\chapter{Objetivos}

\begin{FraseCelebre}
\begin{Frase}
...
\end{Frase}
\begin{Fuente}
...
\end{Fuente}
\end{FraseCelebre}


%-------------------------------------------------------------------
\section{Motivaciones}
%-------------------------------------------------------------------
Nuestra elecci�n por este trabajo es debido a que despues de cursar las signaturas de inteligencia artificial durante el grado hab�an despertado nuestro interes en este campo, y con el trabajo podriamos profundizar a�n m�s. Tambi�n el hecho de tener el hobby de la lectura nos parec�a un reto interesante conseguir hacer un generador de historias con el poder entretener a cualquier usuario, adem�s de hacerle sentir parte fundamental en la historia.

Por �ltimo el reto de ser un proyecto tan amplio es una motivaci�n extra ya que no pone barreras a poder incluir cualquier idea que tengamos.

%-------------------------------------------------------------------
\section{Objetivos}
%-------------------------------------------------------------------

Nuestro principal objetivo es el de desarollar una aplicaci�n capaz de generar historias centradas en los personajes mediante simulaciones. Estas simulaciones de historias queremos que tengan un bajo factor de repetici�n y as� conseguir que cada historia sea totalmente "�nica". Adem�s para poder tener mayor complicidad con el usuario/lector, tendr� la opci�n de crear los personajes que particiaran en la historia.

Todos los personajes que se utilizan en las simulaciones son agentes, que trataremos a trav�s de un Sistema Multiagentes. Cada personaje tiene un objetivo que cumplir y que nos servir� como gui�n en nuestra historia, para poder simular estos objetivos utilizaremos un planificador.

Al tener un trabajo previo, es posible que haya reutilizaci�n de c�digo, todo lo reutilizado se ve en el punto 4 en el apartado de reutilizaci�n. Aunque hay algo que si queremos diferenciar desde el principio y es que ellos en los objetivos dicen "que al aumentar el trabajo se observarian variaciones en la historia, aunque el fin fuese el mismo", y nosotros queremos evitar eso.

Por ello los objetivos que nos marcamos son:
\begin{itemize}
	\item Dise�ar un generador de historias, en el cual cada historia sea "�nica" y no tenga un final predefinido. Para ello:
	\begin{itemize}
		\item Haremos que todos los personajes puedan realizar acciones.
		\item Ampliaremos el mapa.
		\item Introduciremos objetos.
		\item Introduciremos NPC.
	\end{itemize}
	\item Dise�ar una interfaz interactiva con la que queremos sumergir al usuario/lector en la historia.
	\item Introducir replanificaciones en la historia cuando ocurran eventos significativos.
	\item Crear archivos de configuraci�n para favorecer a la aleatoriedad de las historias, guardando el mapa, los objetos, los NPCs.
	\item Crear archivos de configuraci�n que nos permita cambiar el �mbito de las historias generadas.
\end{itemize}

%-------------------------------------------------------------------
\section{Reutilizaci�n}
%-------------------------------------------------------------------
Partimos de un trabajo reducido en el que siempre tenia una �nica introducci�n(El drag�n secuestra a la princesa), con el mismo nudo(el rey contrata caballero) en bucle si se daba la condici�n(drag�n mata caballero) y dos desenlaces (Un caballero salva a la princesa y se convierte en h�roe o el rey se quedaba sin dinero para contratar caballeros y el drag�n se queda con la princesa), esto dista de la idea que tenemos nosotros para el trabajo de generar diversidad en las historias.
