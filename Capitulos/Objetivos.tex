%---------------------------------------------------------------------
%
%                          Cap�tulo 3
%
%---------------------------------------------------------------------
\chapter{Objetivos}

\begin{FraseCelebre}
\begin{Frase}
\end{Frase}
\begin{Fuente}
\end{Fuente}
\end{FraseCelebre}


%-------------------------------------------------------------------
\section{Motivaciones}
%-------------------------------------------------------------------
(necesidades encontradas)

%-------------------------------------------------------------------
\section{Objetivos}
%-------------------------------------------------------------------

Nuestro principal objetivo es el de desarrollar una aplicaci�n capaz de generar historias centradas en los personajes mediante simulaciones. Estas simulaciones de historias queremos que tengan un factor de repetici�n bajo y as� conseguir que cada historia sea totalmente distintiva, "�nica". Adem�s para poder tener mayor complicidad con el usuario/lector, tendr� la opci�n de crear y configurar ciertas parte de la historia, como los personajes que act�an en ella, el nombre de los lugares donde esta transcurre, que tipos de monstruos quiere que aparezcan en su mundo... Pero ya entraremos en eso mas adelante.

Todos los personajes que se utilizan en las simulaciones son agentes, que trataremos a trav�s de un Sistema Multiagentes. Cada personaje tiene unos  objetivos que intentar cumplir y la narraci�n de todo el trascurso de acciones desde su creaci�n hasta alcanzar su objetivo o morir en el intento nos proporcionar� fragmentos de la historia generada. Para poder saber cual es el mejor camino para alcanzar los objetivos y evitar que nuestra historia pierda inter�s utilizaremos un planificador.

Al tener un trabajo previo, es posible que haya reutilizaci�n de c�digo,  en el apartado de reutilizaci�n. Aunque hay algo que si queremos diferenciar desde el principio y es que ellos en los objetivos dicen "que al aumentar el trabajo se observar�an variaciones en la historia, aunque el fin fuese el mismo", y nosotros queremos evitar eso.

Por ello los objetivos que nos marcamos son:
\begin{itemize}
	\item Dise�ar un generador de historias, en el cual cada historia sea "�nica" y no tenga un final predefinido. Para ello:
	\begin{itemize}
		\item Ampliaremos el mapa.
		\item Todos los personajes principales podr�n realizar acciones.		
		\item Introduciremos objetos.
		\item Introduciremos PNJ.
	\end{itemize}
	\item Dise�ar una interfaz interactiva con la que queremos sumergir al usuario/lector en la historia.
	\item Introducir replanificaciones en la historia cuando ocurran eventos significativos.
	\item Utilizar archivos de configuraci�n para favorecer a la aleatoriedad de las historias, guardando el mapa, los objetos, los PNJ's.
	\item Utilizar archivos de configuraci�n que nos permita cambiar el �mbito de las historias generadas.
\end{itemize}

