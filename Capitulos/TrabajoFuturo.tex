%---------------------------------------------------------------------
%
%                          Cap�tulo 13
%
%---------------------------------------------------------------------
\chapter{Trabajo Futuro}

\begin{FraseCelebre}
\begin{Frase}
- Pero Doc, �has construido una m�quina del tiempo con un DeLorean?
-En mi opini�n, si vas a hacer algo como esto, hazlo con estilo.
\end{Frase}
\begin{Fuente}
Regreso al futuro
\end{Fuente}
\end{FraseCelebre}


Con el fin de progresar y ampliar las funcionalidades actuales del sistema se proponen varios campos de investigaci�n para mejorar el mismo.

\section{Personalidad y complejidad de los personajes}

Uno de los aspectos que m�s ha influido en aumentar la variedad de las historias generadas ha sido la inclusi�n de nuevas acciones disponibles. Cuanto mayor sea la cantidad de acciones que pueden realizar los personajes mediante su ejecuci�n y existan acciones m�s at�micas, se podr�n realizar acciones a�n m�s complejas.

Agregar a los personajes aspectos como la personalidad (miedosos, valientes, introvertidos...) y que este aspecto influya tanto en las frases que dicen los personajes como en la probabilidad de realizar con �xito ciertas acciones.

Los atributos y razas como factor aleatorizante en el transcurso de las historias abre un amplio abanico de posibilidades. Incorporar al sistema eventos o limitaciones ligados a estas caracter�sticas enriquecer�a enormemente al sistema.

\section{Estabilidad}

El dr�stico aumento en la capacidad de configuraci�n del sistema nos ha obligado a perder cierto control en la cohesi�n de las diferentes fuentes que le proporcionan datos a la aplicaci�n. Para hacer que el sistema sea m�s sencillo y c�modo de usar por cualquier usuario, eventualmente surgir� la necesidad de tener un m�dulo o herramienta extra la correcta cohesi�n y consistencia entre las diferentes fuentes de entrada de datos.


\section{Interfaz Gr�fica de Usuario}

La interfaz gr�fica actual que usa el sistema es lo suficientemente sencilla para dejar incorporar nuevos personajes a la historia en ejecuci�n pero se ve muy limitada para otras funciones. Extender la implementaci�n de la interfaz para que funcione como cuadro de mandos y permita al usuario introducir los datos iniciales para el sistema en lugar de tener que preparar los archivos XML con antelaci�n facilitar�a el uso de la aplicaci�n.

Disponer de acciones como pausar o poder modificar el estado actual de un personaje mientras la historia se est� generando, as� como aumentar la interactividad general del usuario contra el sistema.

\section{Agentes}

Actualmente, la mayor�a de los agentes pasan buena parte del tiempo esperando a que ocurra alg�n evento, mientras esto pasa, los agentes no realizan ninguna acci�n. Incorporar comportamientos a los agentes para minimizar el tiempo que pasan ociosos y que aporten nueva informaci�n a las historias.