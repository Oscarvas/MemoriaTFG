%---------------------------------------------------------------------
%
%                          Cap�tulo 13
%
%---------------------------------------------------------------------
\chapter{Trabajo Futuro}

\begin{FraseCelebre}
\begin{Frase}
- Pero Doc, �has construido una m�quina del tiempo con un DeLorean?
-En mi opini�n, si vas a hacer algo como esto, hazlo con estilo.
\end{Frase}
\begin{Fuente}
Conversaci�n entre Doc y Marty, Regreso al futuro
\end{Fuente}
\end{FraseCelebre}


Con el fin de progresar y ampliar las funcionalidades actuales del sistema se proponen varios puntos de investigaci�n para mejorar el mismo.

\section{Personalidad y complejidad de los personajes}

Uno de los aspectos que m�s ha influido en aumentar la variedad de las historias generadas ha sido la inclusi�n de nuevas acciones disponibles.
Cuanto mayor sea la cantidad de acciones que pueden realizar los personajes durante su ejecuci�n mayor cantidad de historias habr�.
Es por ello fundamental el que existan m�s acciones at�micas, con las que se podr�n realizar acciones a�n m�s complejas.

Dentro del mismo apartado de personajes, agregar aspectos como la personalidad o estados de animo (miedosos, valientes, introvertidos...) permitir� que este aspecto influya tanto en las frases que dicen los personajes como en la probabilidad de realizar con �xito ciertas acciones.

Por �ltimo, los atributos y razas tomados como un factor aleatorizante en el transcurso de las historias abre un amplio abanico de posibilidades.
Incorporar al sistema eventos o limitaciones ligados a estas caracter�sticas enriquecer�a enormemente al sistema.

\section{Estabilidad}
El aumento en la capacidad de configuraci�n del sistema ha obligado a perder control autom�tico en la cohesi�n de las diferentes fuentes que proporcionan datos a la aplicaci�n.
Para hacer que el sistema sea m�s robusto surgir� la necesidad de tener herramienta encargada de la correcta cohesi�n y consistencia entre las diferentes fuentes de todo el SCD.
De esta forma se conseguir� aumentar la sencillez y comodidad a la hora de configurar y usar el sistema por parte de cualquier usuario.


\section{Interfaz Gr�fica de Usuario}

La interfaz gr�fica actual que usa el sistema es lo suficientemente sencilla para dejar incorporar nuevos personajes a la historia en ejecuci�n pero se ve muy limitada para otras funciones. Extender la implementaci�n de la interfaz para que funcione como cuadro de mandos y permita al usuario introducir los datos iniciales para el sistema en lugar de tener que preparar los archivos XML con antelaci�n facilitar�a el uso de la aplicaci�n.

As� como disponer de acciones que permitan pausar o poder modificar el estado actual de un personaje mientras la historia se est� generando permitir�a aumentar la interactividad general del usuario contra el sistema.

\section{Agentes}

Actualmente, la mayor�a de los agentes pasan buena parte del tiempo esperando a que ocurra alg�n evento, mientras esto pasa, los agentes no realizan ninguna acci�n. Incorporar comportamientos a los agentes para minimizar el tiempo que pasan ociosos y que aporten nueva informaci�n a las historias.

\section{Hilo desencadenante de la historia}
En el momento actual el desencadenante de la historia siempre es que el Secuestrador rapta a la Princesa y su Allegado pide ayuda.
Este proyecto se ha enfocado ha generar multitud de historias a partir de ese momento.
Por esto queda pendiente cambiar el desencadenante de la historia para conseguir que adem�s de generar historias diversas las har�a impredecibles desde el principio.