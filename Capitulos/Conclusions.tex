%---------------------------------------------------------------------
%
%                          Capítulo 14
%
%---------------------------------------------------------------------
\chapter*{Conclusions}

\begin{FraseCelebre}
\begin{Frase}
\end{Frase}
\begin{Fuente}
\end{Fuente}
\end{FraseCelebre}

%-------------------------------------------------------------------
%-------------------------------------------------------------------
\label{cap32:sec:Conclusions}

As has been proved during the development of this research , automated storytelling is a very complex and vast field in which every new aspect on the research adds an almost infinite new number of possibilities. In order to fulfill the objectives of this research, numerous new branches are introduced, causing the risk of not reaching the proposed objectives.


Trying to avoid this problem, this research adds many complex characters to the story, where all his actions have an impact on it. These characters interact between them, causing with their actions the appearance of another character or a change in the way they act. Any of these characters has their unique characteristics, which means that the only possible way of making two identical characters is create them with that purpose.


Similarly, the environment has got a more decisive role in the story, causing that the places a character finally crosses during its execution modify the story because this character will consume the hidden objects in such places or will interact with characters on them .



Thanks to these changes the characters are unpredictable and rarely act in the same way in two consecutive simulations or get the same results.


This way, a highly configurable application where the user is able to tell the story he proposed has been implemented.
On the other hand, we managed to customize the automated storytelling, so the user has no longer to generate different stories on the proposed environment but he is able to define his own environment and tell \emph{his} stories.


So, has the goals been reach??\\
Yes, an automated storytteller capable of generating a lot of varied stories has been developed, with infinite possibilities of personalization which helps a lot generating diversity in the stroytelling.
But the research on this field is far to be over. This same project has open a lot of new variants trying to emulate creativity on an artificial intelligence. Therefore, some of these variants are exposed on the following chapter.



%-------------------------------------------------------------------

