
%---------------------------------------------------------------------
%
%                          Capitulo 1
%
%---------------------------------------------------------------------

\chapter{Personajes}

\begin{FraseCelebre}
\begin{Frase}
...
\end{Frase}
\begin{Fuente}
...
\end{Fuente}
\end{FraseCelebre}

\begin{resumen}
Resumen sobre lo que se va a contar en el capitulo...
\end{resumen}


%-------------------------------------------------------------------
\section{El concepto de clase}
%-------------------------------------------------------------------
\label{cap23:sec:personajes}

Nuestros heroes ya no son los de antes. Por decirlo de alguna manera han madurado, y es su momento de hacer frente a nuevos retos.
Para ello debimos convertir con cuidado a los personajes del antiguo TFG en agentes mas solidos. Esto implicaba que muchas de las acciones que realizaban debian convertirse a comportamientos, integrados dentro de la fantastica interfaz de JADE, y realizar un correcto traspaso de mensajes para que esto funcionase.
Que esperamos de los heroes, las princesas, los monstruos, el rey y los villanos


\section{Creando nuevos personajes}
Nuevos Monstruos
\begin{itemize}
	\item Troll
	\item Serpiente
	\item Fantasma
\end{itemize}

Nuevos Heroes
\begin{itemize}
	\item Mago
	\item Druida
\end{itemize}

\section{Raza y atributos}

La raza es una caracteristica mas que le podemos dar a nuestros personajes.
Como ya se explico en el mapa, la raza determinara en que lugar “nacera” cada personaje, en cierta medida, pues solo aclara la region, no determina en que localizacion especifica de esta, pero su funcionalidad no acaba aqui, pues tambien dara un impulso a los atributos de los personajes.
Con animo de hacer a nuestros personajes aun mas exclusivos, decidimos introducir atributos que representaban su forma fisica y mental. Los atributos que pensamos fueron la vida, la fuerza, la destreza, la inteligencia y la codicia.
Estos atributos representarian modificadores positivos a la hora de enfrentarse a un problema. Por ejemplo, la fuerza seria un multiplicador a la vida, que nos ayuda a enfrentarnos a un monstruo en una batalla, o la codicia representaria el precio que un heroe propondria para contratar sus servicios.

La manera en que los atributos y las razas se relacionan, viene dado porque los personajes de ciertas razas tienen atributos mas acentuados que otros. Esto se representaria como que por ejemplo los personajes de la raza X fuesen especialmente fuertes y los de la raza Y especialmente inteligentes, lo cual no quiere decir sin embargo que no exista un personaje de la raza Y especialmente fuerte que supere a otro de la raza X que no tuvo la misma fortuna.

De igual manera, la clase tambien afectara a estos atributos, ya que un caballero en principio tendria mas opciones de ser fuerte que una princesa, para poder llevar esa flamante armadura tambien hay que echar sus horas en el gimnasio, lo cual nuevamente no implica que no pueda haber una princesa especialmente fuerte, sino que ese caso seria una extra\'na variedad dentro de las multiples historias que podemos formar.

Las razas determinan el sitio donde empiezan (la region en la que empiezan y aleatoriamente un sitio dentro de la region), determinaran en cierta medida los atributos (porque un caballero en principio esta mas fuerte que una princesa, es decir, el tipo de personaje tambien influye en los atributos) y enumerarlas.

\section{Los NPC\'s}
Nuestro numero de historias ya era altamente variado, pero aun asi sentiamos que debiamos proponer mas. En busca de un nuevo aire que diese un nuevo toque a nuestras historias y aumentase en cantidad el numero posible de estas, propusimos la idea de unos personajes simples, parecidos a los agentes objetos que en un principio queriamos evitar, que con una accion sencilla cambiasen ligeramente el curso de la historia o que sencillamente apareciesen en ella dandole un toque divertido a esta.
Para ello nos propusimos crear numerosos personajes entre ellos:

\begin{itemize}
	\item El chaman: guia espiritual, podria guiar a un heroe hasta un objeto magico de gran poder, otorgarle la bendicion de los ancestros o invitarle a una seta sospechosa que menguase su capacidad de lucha.
	\item El Granjero: tiene tierras y ganado. Apareceria en la historia como un personaje campechano que sencillamente desea buena suerte en su viaje al heroe, aunque quiza si ha tenido problemas le pide que rescate a sus ovejas...
	\item El tabernero: un hombre trabajador que conoce las historias que le cuentan los multiples viajeros que han pasado por su establecimiento, quiza solo te invite a una ociosa comida que restaure fuerzas, a cambio de un modico precio, o quiza te cuente la leyenda de una legendaria espada perdido hace tiempo ha en un solitario desierto.
	\item El Chef:un hombre experto en comidas imposibles, aunque a veces algunas no salgan tan bien como desee.
	\item La sastre: una mujer sencilla, que se gana la vida haciendo vestidos sencillos, o quiza una capa de viajante que le recuerde a una princesa su rebeldia.
	\item El herrero: forjador de armaduras, quiza el tenga la clave para que un valiente heroe pueda resistir los golpes de un terrible monstruo.
	\item La bibliotecaria: experta en multiples historias, sabe numerosas leyendas sobre anillos y tomos poderosos que pueden convertir a un sencillo hombre en el heroe que necesita el reino.
\end{itemize}

%-------------------------------------------------------------------
\section*{\ProximoCapitulo}
%-------------------------------------------------------------------
\TocProximoCapitulo

...

% Variable local para emacs, para  que encuentre el fichero maestro de
% compilacion y funcionen mejor algunas teclas rapidas de AucTeX
%%%
%%% Local Variables:
%%% mode: latex
%%% TeX-master: "../Tesis.tex"
%%% End:




