
%---------------------------------------------------------------------
%
%                          Capitulo 1
%
%---------------------------------------------------------------------

\chapter{Multiobjetivo}

\begin{FraseCelebre}
\begin{Frase}
...
\end{Frase}
\begin{Fuente}
...
\end{Fuente}
\end{FraseCelebre}


%-------------------------------------------------------------------
\section{Nada es lo que parece}
%-------------------------------------------------------------------
\label{cap22:sec:objetivos}

Para poder crear historias distintas uno de nuestros principales objetivos era que dos agentes que representaban el mismo tipo de personaje, pudiesen actuar de manera distinta, eligiendo de una manera aleatoria entre un abanico de posibles objetivos cual iba a ser el suyo cada vez.
Afrontar este problema fue un arduo proceso que comenz� cuando tratamos el agente de la princesa. En el TFG anterior, la princesa, era lo que llamamos un agente objeto, un agente que realmente no tenia ninguna funcionalidad, sencillamente era transportada de un lugar a otro, o bien por un drag�n o por un caballero, y no realizaba acciones propias.
Nuestro primer objetivo fue cambiar el agente de la princesa, de manera que a veces decidiese escaparse ella sola si nadie lograba rescatarla. Debido a la construcci�n del anterior TFG, la construcci�n y correcto funcionamiento de un agente clave como la princesa escapaba de nuestro conocimiento sobre agentes, as� que debimos cambiar de objetivo a uno menos ambicioso.


\section{La figura del villano}

Fue aqu� cuando decidimos que el agente que mejor se amoldaba a nuestra b�squeda del agente con varios objetivos era el caballero.
El caballero solo tenia una funcionalidad, rescatar a la princesa, y si fallaba sencillamente se generaba otro caballero que ocupase su lugar.
Para cambiar esto decidimos que no todos los caballeros iban a salvar a la princesa, y as� surgi� el concepto del villano. El villano era un caballero como otro cualquiera, que cuando derrotase al drag�n en vez de salvar a la princesa decidiese secuestrarla y as� otro caballero deb�a venir a salvarla, y con una idea tan simple hab�amos aumentado significantemente nuestra capacidad de crear historias variadas.

Sin embargo esto no fue tan sencillo, nuestro planificador no pod�a soportar tener que elegir aleatoriamente entre dos caballeros distintos, y, bajo nuestro entendimiento del c�digo ya disponible, no se pod�a crear un caballero de manera sencilla que en el punto de inflexi�n para salvar a la princesa decidiese entre rescatarla o volverla a raptar.

Como primera soluci�n decidimos crear al personaje villano, de manera que cuando se creaba un caballero se eleg�a aleatoriamente si este iba a ser caballero o villano desde el principio, as� cuando llegase el momento actuar�an de manera distinta. En cierta medida si, eran dos agentes diferentes, pero a ojos del usuario iban a seguir siendo caballeros, se iban a comportar de distinta manera e iban a generar m�ltiples historias y distintas.

El problema fue que la aplicaci�n no estaba pensada para soportar esto. Las batallas siempre deb�an ser entre un drag�n y un caballero, le costaba adaptar el rol del caballero para que fuese drag�n, el planificador no se entend�a con los cambios? Tras mucho debatir y con ciertos reparos, pues tras cortarle una cabeza a nuestro problema, de esta salieron dos, decidimos que la manera optima de actuar era replanteando el TFG desde cero, siempre partiendo de la base del antiguo, pero reciclando el las funcionalidades de los agentes para que se adecuasen a nuestras necesidades, planteando los personajes y su manera de actuar como nosotros necesit�bamos y replanteando y depurando el c�digo para que, a nuestro entender, estuviese mucho mejor estructurado, de una manera mas l�gica, respetando mejor los usos de los agentes y JADE, haciendo que las acciones fuesen behaviours de JADE y no m�todos sueltos, y en definitiva fuese mas legible para la comprensi�n del programador.

\section{Empezando desde cero}

Rehaciendo el c�digo, abriendo un nuevo proyecto, que queremos, podemos usar, concepto de GUI, mejora visual, ahora lo entendemos, to palante. 
Un nuevo rey, una nueva princesa, el drag�n al uso, el caballero al uso.

Esto nos permiti� satisfacer nuestra necesidad del multiobjetivo, y as� pudimos transformar a nuestros agentes en personajes mas profundos, con una serie de posibles objetivos:
(INTRODUCIR TABLA AQU�, o bien mintiendo y poniendo solo los posibles objetivos de la princesa y caballero o bien introducir la tocho tabla con los distintos objetivos de cada tipo de personaje)

Hasta que por fin, y ahora si, pod�amos tener caballeros que actuasen como villanos y princesas que actuasen y permiti�ndonos finales distintos.


\section{Ahora que ya estamos donde queremos}
Era el momento de ponernos ambiciosos, as� que nos propusimos crear muchos mas personajes, rellenar el mapa con nuevos monstruos, e incluso poner algunos personajes totalmente secundarios que sencillamente le diesen un toque picante a la historia

