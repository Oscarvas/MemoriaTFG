%---------------------------------------------------------------------
%
%                          Cap�tulo 14
%
%---------------------------------------------------------------------
\chapter{Conclusiones}

\begin{FraseCelebre}
\begin{Frase}
\end{Frase}
\begin{Fuente}
\end{Fuente}
\end{FraseCelebre}

%-------------------------------------------------------------------
%-------------------------------------------------------------------
\label{cap32:sec:Conclusiones}
Como bien se ha comprobado  durante el desarrollo de esta investigaci�n, la generaci�n autom�tica de historias es un campo muy complejo y extenso, en el que cualquier nueva vertiente a�ade un abanico casi infinito de posibilidades nuevas para agregar a la investigaci�n. Al querer cumplir los objetivos propuestos para el sistema, la investigaci�n se introduce en numerosas ramas, provocando que se corra el riesgo de abarcar demasiado y no llegar a los objetivos propuestos.

Con el suficiente cuidado para evitar este problema, esta investigaci�n ha conseguido a�adir numerosos personajes complejos a la narraci�n, donde todas sus acciones tienen repercusi�n en la misma. Estos personajes interact�an entre ellos, provocando que las acciones de un personaje desencadenen la aparici�n de otro o provoquen un cambio en su manera de actuar. As� mismo, sus diferentes rasgos y caracter�sticas les hacen �nicos, haciendo que la �nica manera posible de hacer dos personajes iguales es crearlos con ese prop�sito.

De la misma manera, el entorno ha conseguido un papel m�s determinante en la historia, provocando que los lugares que finalmente atraviesa un personaje durante su ejecuci�n modifiquen el resultado de la historia, al consumir el personaje los objetos escondidos en dichos lugares o provocando que se interact�e con los personajes ligados a �stos.

Gracias a estos cambios se consigue que los personajes sean poco predecibles y que muy raras veces act�en de la misma manera en dos simulaciones consecutivas o consigan los mismos resultados.

Tambi�n se ha conseguido una aplicaci�n altamente configurable donde el usuario es capaz de narrar de una manera m�s o menos eficiente la historia que �l a propuesto.
De esta manera, se ha conseguido personalizar la generaci�n autom�tica de historias, de manera que ya no tienen por que generarse diferentes historias sobre el escenario propuesto, si no que el usuario es capaz de definir su propio entorno y centrar la generaci�n de historias autom�tica en torno a \emph{su} historia.

�Se puede decir por tanto que se ha cumplido el objetivo?\\
Si, se ha conseguido crear un generador de historias autom�tico capaz de generar una gran cantidad de historias variadas, con infinitas posibilidades de personalizaci�n que a su vez generan mayor diversidad a la hora de crear historias.
Pero ni mucho menos la investigaci�n sobre la generaci�n autom�tica de historias ha acabado. Esta misma investigaci�n a abierto nuevas variantes para tratar la diversidad en la generaci�n o intentar emular la creatividad en la inteligencia artificial. Algunas de estas variantes son expuestas en el siguiente cap�tulo.


%-------------------------------------------------------------------

