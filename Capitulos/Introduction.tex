%---------------------------------------------------------------------
%
%                          Cap�tulo 2
%
%---------------------------------------------------------------------

\chapter*{Introduction}

\begin{FraseCelebre}
\begin{Frase}
\end{Frase}
\begin{Fuente}
\end{Fuente}
\end{FraseCelebre}

Computational Storytelling raises some problems, which are summarized in how precisely can imitate the artificial intelligence the human intelligence.
More specifically, the human characteristic that stands out when creating a story is creativity. Creativity is an abstract concept difficult to explain. According to RAE it is ``the ability to create'', which means that needs to be original. These requires for creativity to innovate and recycle itself every few years. It is, therefore, a difficult feature to find among people, since not everyone has the ability to be creative and, much less, to tell an interesting story.

Is, therefore, creativity a feature that can be given to an artificial intelligence?
Yes and no. The artificial intelligence is not capable by itself to create, because it is always settled and limited by the features provided by his programmer. Creativity can not be achieved on the artificial intelligence, but it can be simulated. Through simulating creativity, we are able to create different stories with the same software. And to do so we have lot of options.

The first way to simulate creativity is by making completely different stories, like, for example, proposing different places in which the story takes place, such as a fantastic medieval environment or a Western scenario. It can also presents totally different characters. Following the previous example, in one story you will got knights and dragons and in the other one sheriffs and bandits. Thanks to this, the number of stories you can get increases considerably.


However, this is not the only way to achieve this, you can also seek that two stories with the same assumptions turn out to be different.
If two stories have the same characteristics, the only way to make them develop into two different stories is to find a moment in which them diverge. We currently believe that this moment should happen when the characters conflict in the story. This way, considering the features of each character or just if they appear on the right time, the story will take one way or another.


Given these possibilities, two solutions are proposed. The first one is to expose a highly configurable application where the user can choose the details of the story, as well as the names of the characters or the details of the world designed. Similarly, the possibility of having a wide range of possible characters is also an option, dividing these characters in several classes, so that different characters will behave in different ways and produce new stories.

The second way is to make every element of the application unique. To do this, characteristics are given to the characters in order to make them different from their peers, as well as the possibility of getting objects that may help them in the different conflicts that will involved them. These changes introduce conflicts between the characters, with a battle as the most typical example, making them dice rolls like in role games. These dice rolls are modified by these qualities. In this way, some characters are more likely to win than others and the factor of ``luck'' always gives a special feature to every story.

In order to get this idea, the use of Intelligent Agents is needed. These software units with an advanced artificial intelligence can be proposed with a goal that they will always try to reach. These Intelligent Agents will play the role of the different characters. For these Agents to understand their targets and interpret them rightly in the world they simulate to live in, a planner is required This planner is a tool able to propose to each Agent the specific steps that they should follow in order to fulfill their objectives.
This leads to the agents having conflicts between them, because sometimes the goals of an Agent are opposite to the goals of one another. To solve this problem, a simulator is needed, a tool able to simulate a battle, for example, between different characters to see which shall become the winner so he can kep on with his plan to succeed on its objective.


In an easier way, imagine that the story is like rolling a film. The agents represent the characters or actors of the movie, each with its own interests and motivations. The planner will be the writing team, deciding which way each character should take in order to fulfill its role in the story. And finally, the simulation will act as the manager, deciding on important scenes which character will go forward in the story and which will be excluded (like in a battle, in each simulation the winner may be different). These way the story focus again on the characters that remained available. The simulation will ask the writing team, the planner, to determine what the next course will be, until once again it is needed to re-simulate.


Further on, the characteristics of the project are explained, as well as all the necessary tools that we need to understand its functioning and the qualities we should provide hoping that it would be able to meet our expectations. To do so, this memory is structured as follows:


\section*{Document General Structure}
\begin{itemize}
	\item \textbf{Cap�tulo 1: Introduccion.} 
%	\item \textbf{Cap�tulo 1: Introduction.}
	
	In this chapter the investigation is introduced.	
	
	\item \textbf{Cap�tulo 2: Estado de la Cuesti�n. }
	
	Chapter in which the technologies used on the research are explained, as well as how they work.
	\item \textbf{Cap�tulo 3: Objetivos. }
%	\item \textbf{Cap�tulo 3: Objectives. }
	
	Chapter that indicates specifically the objectives set for this research. 
	
%	\item \textbf{Cap�tulo 4: First Steps.}	
	\item \textbf{Cap�tulo 4: Primeros Pasos.} 
	
	Chapter that describes the experience in the early stages of the research and the problems founded, making possible to understand the development of the investigation.	
	
%	\item \textbf{Cap�tulo 5: Storyteller Arquitecture.}
	\item \textbf{Cap�tulo 5: Arquitectura del Generador de historias.}
	
	Chapter that provides an advanced knowledge of the system architecture.
	
%	\item \textbf{Cap�tulo 6: The Map.}
	\item \textbf{Cap�tulo 6: Mapa.}
	
	Chapters that describes the world where the story takes place, write out as a Map, as well as its design and implementation. 
	
%	\item \textbf{Cap�tulo 7: Characters. }
	\item \textbf{Cap�tulo 7: Personajes. }
	
	
	Chapter where the design and implementation of the different characters in the story is explained, as well all the qualities that define them.
	
%	\item \textbf{Cap�tulo 8: Objects. }	
	\item \textbf{Cap�tulo 8: Objetos. }
	
	Chapter in wich the objects development and how different characters can use them is detailed.

%	\item \textbf{Cap�tulo 9: Simulations. }
	\item \textbf{Cap�tulo 9: Simulaciones. }
	
	Chapter in which different output examples of the created software are exposed.	
 
%	\item \textbf{Cap�tulo 10: Individual Work. }
	\item \textbf{Cap�tulo 10: Trabajo Individual. }
	
	Chapter in which is detailed the work performed by each team member on the research.
%	\item \textbf{Cap�tulo 11: Conclusions. }
	\item \textbf{Cap�tulo 11: Conclusiones. }
		
	Chapter in which the different conclusions which has led the investigation are set.
	
%	\item \textbf{Cap�tulo 12: Future Work. }
	\item \textbf{Cap�tulo 12: Trabajo Futuro. }
	
	Chapter in which some ideas to continue working on the same research get proposed. 

\end{itemize}
