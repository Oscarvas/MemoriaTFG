%---------------------------------------------------------------------
%
%                          Cap�tulo 9 BIS
%
%---------------------------------------------------------------------

\chapter{Dise�o e Implementaci�n de los Atributos y Razas}

\begin{FraseCelebre}
\begin{Frase}
\end{Frase}
\begin{Fuente}
\end{Fuente}
\end{FraseCelebre}

%-------------------------------------------------------------------

\section{Dise�o de los Atributos y las Razas}

%Como ya se explico en el mapa, la raza determinara en que lugar ?nacer�? cada personaje, en cierta medida, pues solo aclara la regi�n, no determina en que localizaci�n especifica de esta, pero su funcionalidad no acaba aqu�, pues tambi�n dar� un impulso a los atributos de los personajes.
Con animo de hacer a nuestros personajes aun mas exclusivos, decidimos introducir atributos que representaban su forma f�sica y mental. Los atributos que pensamos fueron la vida, la fuerza, la destreza, la inteligencia y la codicia.
Estos atributos representar�an modificadores positivos a la hora de enfrentarse a un problema. Por ejemplo, la fuerza seria un multiplicador a la vida, que nos ayuda a enfrentarnos a un monstruo en una batalla, o la codicia representar�a el precio que un h�roe propondr�a a un rey para que este contratase sus servicios.

De la misma forma, se propuso el concepto de raza, que es una caracter�stica mas que le podemos dar a los personajes protagonistas.
Esta nueva caracter�stica modificar� seg�n ciertos multiplicadores los atributos que un personaje posee, haciendo as� que dos h�roes de razas distintas muy probablemente tengan atributos distintos.

La manera en que los atributos y las razas se relacionan, viene dado porque los personajes de ciertas razas tienen atributos mas acentuados que otros. Esto se implicar�a que, por ejemplo, los personajes de la raza X fuesen especialmente fuertes y los de la raza Y especialmente inteligentes, lo cual no quiere decir sin embargo que no exista un personaje de la raza Y especialmente fuerte que supere a otro de la raza X que no tuvo la misma fortuna.

De igual manera, la clase tambi�n afectara a estos atributos, ya que un caballero en principio tendr�a mas opciones de ser fuerte que una princesa, para poder llevar esa flamante armadura tambi�n hay que echar sus horas en el gimnasio, lo cual nuevamente no implica que no pueda haber una princesa especialmente fuerte, sino que ese caso seria una extra�a variedad dentro de las m�ltiples historias que podemos formar.



%-------------------------------------------------------------------
\section{Implementaci�n de los Atributos y las Razas}

%-------------------------------------------------------------------