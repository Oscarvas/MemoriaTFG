%---------------------------------------------------------------------
%
%                          Cap�tulo 11
%
%---------------------------------------------------------------------

\chapter{Narrativa del sistema}

\begin{FraseCelebre}
\begin{Frase}
\end{Frase}
\begin{Fuente}
\end{Fuente}
\end{FraseCelebre}

\label{cap:Frases}

Este capitulo es fundamental dentro de un sistema generador de historias narrativas, ya que es el punto de conexi�n entre la aplicaci�n y el usuario. 

\section{Dise�o de la narrativa}
Si se observa dentro del campo de la generaci�n de narrativa computacional la mayor�a de sistemas est�n ligados a un �nico tipo de historias, como se puede observar en el punto \ref{cap:EstadoDeLaCuestion:Narrativa}. Continuando el afan de crear un sistema altamente configurable, y tambi�n de poder dar la oportunidad a un escritor profesional crear una mejor narrativa para el sistema sin tener que entrar dentro de los apartados m�s t�cnicos, se dise�o el apartado narrativo, en su mayor�a, de forma externa a la aplicaci�n. Hay conectores ling��sticos que se han tenido que quedar dentro de la aplicaci�n.

La importancia de este dise�o es de gran valor ya que permite cambiar de forma m�s sencilla cambiar tanto el �mbito de la historia, ser�a posible pasar de una historia fant�stica medieval a una futur�stica, como poder generar la historia en otros idiomas. 

\section{Implementaci�n de la narrativa}

\subsection{Carga en el sistema}

\subsection{Interacci�n con las clases }
