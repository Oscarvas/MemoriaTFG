%---------------------------------------------------------------------
%
%                          Cap�tulo 9
%
%---------------------------------------------------------------------

\chapter{Dise�o e Implementaci�n de los PNJ's}

\begin{FraseCelebre}
\begin{Frase}
\end{Frase}
\begin{Fuente}
\end{Fuente}
\end{FraseCelebre}

\textit{``Los PNJ's son personajes del juego que son controlados por el sistema y que son capaces de interactuar con el avatar del jugador intercambiar art�culos e informaci�n''} \cite{GomezGauchiaPeinadoTIDSE06}. En el sistema hemos adaptado estos personajes de los juegos a la narrativa para conseguir mayor diversidad de historias generadas.
%-------------------------------------------------------------------

\section{Dise�o de los PNJ's}
Nuestro numero de historias ya era altamente variado, pero aun as� sent�amos que deb�amos proponer mas. En busca de un nuevo aire que diese un toque distinto a nuestras historias y aumentase en cantidad el numero posible de estas, propusimos la idea de unos personajes simples, parecidos a los agentes objetos que en un principio quer�amos evitar, que con una acci�n sencilla cambiasen ligeramente el curso de la historia o que sencillamente apareciesen en ella d�ndole un toque divertido a �sta.

Para ello nos propusimos crear numerosos personajes, que tuviesen como peculiaridad el interactuar con personajes principales cuando �stos pasaban por las localizaciones en las que estaban situados. En ese momento se producir�a un peque�o di�logo entre ambos personajes, otorg�ndole el PNJ un objeto, \ref{cap11:sec:ObjetosPNJ}, al personaje principal. A este aspecto dentro de la historia lo denominar�amos como ``bufo''. �ste "bufo" bien pod�a ser un objeto que se consum�a en el momento o una bendici�n, que afectaba al personaje modificando sus atributos, ya fuese de manera positiva por que se hab�a encontrado con un PNJ amigable o de manera negativa si �ste era perverso. 

Para terminar de definir los PNJ's pensamos que lo que mas encajaba dentro de nuestra propuesta de dise�o es que �stos tuviesen un oficio. De esta manera, al crear los PNJ's propusimos que para definir el objeto con el que interactuaban lo suyo era clasificarlos por su oficio, de manera que decidimos que nuestros PNJ's fuesen herreros, enfermeras o sastres, que viv�an en los lugares relacionados con su profesi�n tal y como se comento en el capitulo de mapa\ref{cap5:sec:LocalizacionesPNJ} vease la herrer�a, el hospital o la sastrer�a y proporcionasen este buffo con algo relacionado a su campo, por ejemplo el herrero podr�a forjar una espada mejor o reestablecer las abolladuras de la armadura del personaje, lo que se acabar�a representando como una mejora en un atributo del propio personaje, como puede ser la fuerza con la espada o la vida con la armadura. 
De la misma manera, y para que entendamos el concepto de PNJ mal�volo, el personaje podr�a cruzarse con un cham�n, que le da una p�zima de setas ''bendecidas por los ancestros��, donde en ciertas simulaciones aumentaran los atributos del personaje, y en otras los disminuir�n.



%-------------------------------------------------------------------
\section{Implementaci�n de los PNJ's}

%-------------------------------------------------------------------