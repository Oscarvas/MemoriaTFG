%---------------------------------------------------------------------
%
%                          Cap�tulo 9
%
%---------------------------------------------------------------------

\chapter{Dise�o e Implementaci�n de los PNJ's}

\begin{FraseCelebre}
\begin{Frase}
\end{Frase}
\begin{Fuente}
\end{Fuente}
\end{FraseCelebre}

\textit{``Los PNJ's son personajes del juego que son controlados por el sistema y que son capaces de interactuar con el avatar del jugador intercambiar art�culos e informaci�n''} \cite{GomezGauchiaPeinadoTIDSE06}. En el sistema hemos adaptado estos personajes de los juegos a la narrativa para conseguir mayor diversidad de historias generadas.
%-------------------------------------------------------------------

\section{Dise�o de los PNJ's}
Nuestro numero de historias ya era altamente variado, pero aun as� sent�amos que deb�amos proponer mas. En busca de un nuevo aire que diese un toque distinto a nuestras historias y aumentase en cantidad el numero posible de estas, propusimos la idea de unos personajes simples, parecidos a los agentes objetos que en un principio quer�amos evitar, que con una acci�n sencilla cambiasen ligeramente el curso de la historia o que sencillamente apareciesen en ella d�ndole un toque divertido a �sta.

Para ello nos propusimos crear numerosos personajes, que tuviesen como peculiaridad el interactuar con personajes principales cuando �stos pasaban por las localizaciones en las que estaban situados. En ese momento se producir�a un peque�o di�logo entre ambos personajes, otorg�ndole el PNJ un objeto, \ref{cap11:sec:ObjetosPNJ} al personaje principal. A este aspecto dentro de la historia lo denominar�amos como ``bufo''. �ste "bufo" bien pod�a ser un objeto que se consum�a en el momento o una bendici�n, que afectaba al personaje modificando sus atributos, ya fuese de manera positiva por que se hab�a encontrado con un PNJ amigable o de manera negativa si �ste era perverso. 

ASI NO.
\begin{itemize}
	\item El chaman: gu�a espiritual, podr�a guiar a un h�roe hasta un objeto m�gico de gran poder, otorgarle la bendici�n de los ancestros o invitarle a una seta sospechosa que menguase su capacidad de lucha.
	\item El Granjero: tiene tierras y ganado. Aparecer�a en la historia como un personaje campechano que sencillamente desea buena suerte en su viaje al h�roe, aunque quiz� si ha tenido problemas le pide que rescate a sus ovejas...
	\item El tabernero: un hombre trabajador que conoce las historias que le cuentan los m�ltiples viajeros que han pasado por su establecimiento, quiz� solo te invite a una ociosa comida que restaure fuerzas, a cambio de un m�dico precio, o quiz� te cuente la leyenda de una legendaria espada perdido hace tiempo ha en un solitario desierto.
	\item El Chef:un hombre experto en comidas imposibles, aunque a veces algunas no salgan tan bien como desee.
	\item La sastre: una mujer sencilla, que se gana la vida haciendo vestidos sencillos, o quiz� una capa de viajante que le recuerde a una princesa su rebeld�a.
	\item El herrero: forjador de armaduras, quiz� el tenga la clave para que un valiente h�roe pueda resistir los golpes de un terrible monstruo.
	\item La bibliotecaria: experta en m�ltiples historias, sabe numerosas leyendas sobre anillos y tomos poderosos que pueden convertir a un sencillo hombre en el h�roe que necesita el reino.
\end{itemize}

%-------------------------------------------------------------------
\section{Implementaci�n de los PNJ's}

%-------------------------------------------------------------------