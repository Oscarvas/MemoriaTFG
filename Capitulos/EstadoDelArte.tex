%---------------------------------------------------------------------
%
%                          Cap�tulo 2
%
%---------------------------------------------------------------------

\chapter{Estado del Arte}

\begin{FraseCelebre}
\begin{Frase}
\end{Frase}
\begin{Fuente}
\end{Fuente}
\end{FraseCelebre}

En este capitulo hablaremos del trabajo de investigaci�n que ha sido llevado a cabo durante la realizaci�n de nuestro proyecto, y explicaremos las bases que hemos utilizado durante su realizaci�n.

\section{Multiagentes}
%Revisar, solo esta copiado directamente del drive
%Para nosotros un agente es una unidad software con capacidad de actuar en funci�n y en conjunto o en contra de otras unidades software, como pueden ser otros agentes. Para ello est�n dotados de una falsa independencia, donde, siempre siguiendo las leyes de la programaci�n, son capaces de actuar de una u otra forma.
\section{Planificaci�n}

\section{Sistemas}

\subsection[Proyecto anterior]{Generador de historias basado en agentes}
Este fue el primer sistema que estudiamos para empezar a plantear nuestro proyecto. Es proyecto de fin de grado de unos alumnos de la Facultad de Inform�tica de la Universidad Complutense de Madrid de la promoci�n de 2014. Este proyecto nos ha servido como base para empezar a desarrollar nuestro propio proyecto ya que utilizan un sistema multiagentes como es Jade y usan un planificador para ayudar a las simulaciones. Ha sido muy importante estudiar tanto su memoria como su c�digo fuente para poder diferenciar bien su trabajo del nuestro.

Partimos de un trabajo reducido en el que siempre ten�amos un mismo inicio (El drag�n secuestra a la princesa), con un nudo(el rey contrata caballero) en bucle si se cumpl�a una condici�n(drag�n mata caballero) y dos desenlaces (Un caballero salva a la princesa y se convierte en h�roe o el rey se quedaba sin dinero para contratar caballeros y el drag�n se queda con la princesa), esto dista de la idea que tenemos para generar historias diversas y cada vez que se ejecute la aplicaci�n genere una historia �nica.


