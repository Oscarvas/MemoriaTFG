%---------------------------------------------------------------------
%
%                      resumen.tex
%
%---------------------------------------------------------------------
%
% Contiene el cap�tulo del resumen.
%
% Se crea como un cap�tulo sin numeraci�n.
%
%---------------------------------------------------------------------

\chapter{Resumen}
\cabeceraEspecial{Resumen}

\begin{FraseCelebre}
\begin{Frase}
Hace frio ahi fuera
\end{Frase}
\begin{Fuente}
Tabernero del Hearthstone: Heroes of Warcraft
\end{Fuente}
\end{FraseCelebre}

%sobre que hablamos
La generaci�n de historias computacional es un camino que ya empez� a recorrerse hace mucho tiempo. El objetivo de este campo es de dotar de creatividad a la inteligencia artificial, intentando de esta manera que sean capaces de emular la creatividad humana a la hora de la narraci�n. Si bien es cierto que es un campo sobre el que ya se ha debatido y trabajado de innumerables maneras, es igual de cierto que todav�a no se ha llegado a una soluci�n satisfactoria.
Intentando aportar su propio granito de arena, este proyecto trata de crear personajes realmente independientes y hasta cierto punto imprevisibles, capaces de controlar de manera aut�noma sus propias decisiones e intentando dotarles de una mayor independencia a la hora de actuar con el fin de crear historias mas variadas.

%que campo vamos a abarcar en especifico
Para cumplir este prop�sito existe un campo concreto que es la \texttt{Generaci�n de historias a trav�s de Agentes Inteligentes}. Estos agentes representan unidades software con una inteligencia avanzada y son capaces de comunicarse entre ellos, tratando de cumplir objetivos espec�ficos que se le presentar�n a lo largo de la historia para lo cual compiten contra otros agentes y a trav�s de simulaciones, se deciden los resultados que van generando los conflictos en la historia.

%a grandes rasgos que hemos hecho y por que
Para poder producir esta idea, se desarrolla una aplicaci�n en \texttt{Java} que mediante un sistema de \texttt{Agentes Inteligentes}, apoyados por la plataforma \texttt{JADE}, consiguen a trav�s de la planificaci�n de sus distintos objetivos, lograda gracias a interactuar con un \texttt{Planificador} externo, crear por medio de \texttt{simulaciones} un numero relevante de historias variadas.

Siguiendo este proceso, ya explorado en anteriores investigaciones de esta facultad, se propone un sistema de generaci�n de historias mas variado capaz de crear un numero mayor de personajes, as� como de personajes mas completos y un mundo mas complejo.
A trav�s de este sistema, las historias se producen en mundos variados, con numerosos personajes capaces de actuar de manera distinta en cada simulaci�n. Estos personajes cuentan con sus propias normas as� como sus rasgos y caracter�sticas que los definen como �nicos, son tambi�n capaces de interactuar con objetos y tienen un sistema mas variado de generaci�n de frases que se reflejan en la narraci�n, favoreciendo enormemente la diversidad a la hora de generar historias.


%mini conclusion de resultados

A continuaci�n, expondremos las pruebas realizadas y las soluciones obtenidas en el desarrollo de la aplicaci�n y procederemos a explicar el plan de trabajo, documentando las bases t�cnicas de las tecnolog�as usadas as� como detallando en profundidad el trabajo realizado, con el objetivo de poder continuar este trabajo de investigaci�n o aportar nuestra experiencia para futuras investigaciones relacionadas con el tema.

%En este proyecto queremos continuar el trabajo de fin de grado 'Generador de historias basado en agentes' comenzado por nuestros compa�eros en 2014.
%Nuestra intenci�n a la hora de ampliar dicho trabajo es la de generar creatividad computacional de forma que  las historias que generemos se�n m�s  cercanas a la creatividad humana. Para realizar este trabajo utilizaremos como base de nuestro proyecto los conocimientos que han recogido nuestros compa�eros en el trabajo previamente realizado.

%El trabajo realizado en este proyecto est� centrado en la generaci�n de historias basadas en los personajes. Por ello introduciremos caracter�sticas y rasgos a los personajes adem�s de a�adir  elementos en el mundo para que interactuen los personajes y as� poder crear historias diferentes en cada una de las simulaciones. 
%Durante estas  simulaciones interactuaremos con un planificador externo para que nos de el mejor plan posible para que cada personaje cumpla su objetivo.

\section*{Palabras Claves}
Sistema generador de narrativa
Sistema multiagente
Simulaci�n
Planificaci�n

\endinput
% Variable local para emacs, para  que encuentre el fichero maestro de
% compilaci�n y funcionen mejor algunas teclas r�pidas de AucTeX
%%%
%%% Local Variables:
%%% mode: latex
%%% TeX-master: "../Tesis.tex"
%%% End:
