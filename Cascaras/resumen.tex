%---------------------------------------------------------------------
%
%                      resumen.tex
%
%---------------------------------------------------------------------
%
% Contiene el cap�tulo del resumen.
%
% Se crea como un cap�tulo sin numeraci�n.
%
%---------------------------------------------------------------------

\chapter{Resumen}
\cabeceraEspecial{Resumen}

\begin{FraseCelebre}
\begin{Frase}
Hace frio ahi fuera
\end{Frase}
\begin{Fuente}
Tabernero del Hearthstone: Heroes of Warcraft
\end{Fuente}
\end{FraseCelebre}

En este proyecto queremos ampliar la creatividad computacional del proyecto anteriormente realizado.Nos vamos a concentrar en el otorgar una gran aleatoriedad en la narraci�n de las historias. Para ello cada agente tendr� la capacidad de realizar un diverso n�mero de acciones lo que permitir� que el curso de la historia no sea igual siempre.

\endinput
% Variable local para emacs, para  que encuentre el fichero maestro de
% compilaci�n y funcionen mejor algunas teclas r�pidas de AucTeX
%%%
%%% Local Variables:
%%% mode: latex
%%% TeX-master: "../Tesis.tex"
%%% End:
