%---------------------------------------------------------------------
%
%                      resumen.tex
%
%---------------------------------------------------------------------
%
% Contiene el cap�tulo del resumen.
%
% Se crea como un cap�tulo sin numeraci�n.
%
%---------------------------------------------------------------------

\chapter{Resumen}
\cabeceraEspecial{Resumen}

\begin{FraseCelebre}
\begin{Frase}
Hace frio ahi fuera
\end{Frase}
\begin{Fuente}
Tabernero del Hearthstone: Heroes of Warcraft
\end{Fuente}
\end{FraseCelebre}

En este proyecto queremos continuar el trabajo de fin de grado 'Generador de historias basado en agentes' comenzado por nuestros compa�eros en 2014.
Nuestra intenci�n a la hora de ampliar dicho trabajo es la de generar creatividad computacional de forma que  las historias que generemos se�n m�s  cercanas a la creatividad humana. Para realizar este trabajo utilizaremos como base de nuestro proyecto los conocimientos que han recogido nuestros compa�eros en el trabajo previamente realizado.

El trabajo realizado en este proyecto est� centrado en la generaci�n de historias basadas en los personajes. Por ello introduciremos caracter�sticas y rasgos a los personajes adem�s de a�adir  elementos en el mundo para que interactuen los personajes y as� poder crear historias diferentes en cada una de las simulaciones. 
Durante estas  simulaciones interactuaremos con un planificador externo para que nos de el mejor plan posible para que cada personaje cumpla su objetivo.

\section*{Palabras Claves}
Sistema generador de narrativa
Sistema multiagente
Simulaci�n
Planificaci�n

\endinput
% Variable local para emacs, para  que encuentre el fichero maestro de
% compilaci�n y funcionen mejor algunas teclas r�pidas de AucTeX
%%%
%%% Local Variables:
%%% mode: latex
%%% TeX-master: "../Tesis.tex"
%%% End:
