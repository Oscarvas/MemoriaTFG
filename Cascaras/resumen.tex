%---------------------------------------------------------------------
%
%                      resumen.tex
%
%---------------------------------------------------------------------
%
% Contiene el cap�tulo del resumen.
%
% Se crea como un cap�tulo sin numeraci�n.
%
%---------------------------------------------------------------------

\chapter{Resumen}
\cabeceraEspecial{Resumen}

\begin{FraseCelebre}
\begin{Frase}
Hace fr�o ah� fuera
\end{Frase}
\begin{Fuente}
Tabernero del Hearthstone: Heroes of Warcraft
\end{Fuente}
\end{FraseCelebre}

%sobre que hablamos
La \texttt{Generaci�n de historias computacional} es un camino que empez� a recorrerse a principios de los a�os 70. El objetivo de este campo es el de dotar de una cualidad humana, la creatividad, a una inteligencia artificial con el fin de que sea capaz de reproducir esta capacidad y demostrar que es competente para narrar una historia. Si bien es cierto que es un campo sobre el que ya se ha trabajado y debatido en innumerables ocasiones, es igual de cierto que la soluciones dadas no cubren la soluci�n en su totalidad.
Intentando aportar un granito de arena al problema, este proyecto trata de crear personajes independientes e imprevisibles, en la medida que es esto posible, capaces de controlar de manera aut�noma y personal sus propias decisiones con el fin de crear historias variadas dentro de un mismo entorno.

%que campo vamos a abarcar en especifico
Para cumplir este prop�sito existe un campo concreto que es la \texttt{Generaci�n de historias} a trav�s de \texttt{Agentes Inteligentes}. Estos Agentes representan entidades software con una inteligencia artificial avanzada que les permite percibir su entorno e interactuar con �l, comunic�ndose con otros Agentes mientras tratan de cumplir objetivos espec�ficos que se les presentar�n a lo largo de la historia, los cuales provocaran que surjan conflictos de inter�s entre los distintos Agentes, y mediante simulaciones se deciden los resultados de estos conflictos generando con sus resultados la historia.

%a grandes rasgos que hemos hecho y por que
Para poder producir esta idea, se desarrolla una aplicaci�n en \texttt{Java} que mediante un sistema de \texttt{Agentes Inteligentes}, apoyados por la plataforma \texttt{JADE}, consiguen a trav�s de la planificaci�n de distintos objetivos, lograda gracias a interactuar con un \texttt{Planificador} externo, crear por medio de \texttt{simulaciones} un n�mero relevante de historias variadas.

Siguiendo este proceso, ya explorado en anteriores investigaciones de esta facultad, se propone un sistema de generaci�n de historias variadas capaz de trabajar  con un n�mero mayor de Agentes, as� como de generar personajes m�s profundos y un entorno con mayor complejidad.
%Lo comento porque me parece entrar en detalle pronto
%A mi me parece que esta bien, es mini entrar en detalle pero resumiendo un poco lo que queremos aportar nosotros a la investigacion
A trav�s de este sistema se pretende que las historias se produzcan en mundos variados, con numerosos personajes capaces de actuar de manera distinta en cada simulaci�n. Estos personajes contar�n con sus propias normas as� como sus rasgos y caracter�sticas que los definir�n como �nicos, ser�n capaces de interactuar con objetos y tendr�n un sistema mas variado de generaci�n de frases para reflejarse en la narraci�n, favoreciendo enormemente la diversidad a la hora de generar historias.


%mini conclusion de resultados

A continuaci�n, expondremos las pruebas realizadas y las soluciones obtenidas en el desarrollo de la aplicaci�n y procederemos a explicar el plan de trabajo, documentando las bases t�cnicas de las tecnolog�as usadas as� como detallando en profundidad el trabajo realizado, con el objetivo de poder continuar este trabajo de investigaci�n o aportar nuestra experiencia para futuras investigaciones relacionadas con el tema.

\section*{Palabras Claves}
Generaci�n de historias
Sistema multiagente
Simulaci�n
Planificaci�n

\endinput
% Variable local para emacs, para  que encuentre el fichero maestro de
% compilaci�n y funcionen mejor algunas teclas r�pidas de AucTeX
%%%
%%% Local Variables:
%%% mode: latex
%%% TeX-master: "../Tesis.tex"
%%% End:
