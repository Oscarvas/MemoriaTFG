%---------------------------------------------------------------------
%
%                      resumen.tex
%
%---------------------------------------------------------------------
%
% Contiene el cap�tulo del resumen.
%
% Se crea como un cap�tulo sin numeraci�n.
%
%---------------------------------------------------------------------

\chapter{Abstract}
\cabeceraEspecial{Abstract}

\begin{FraseCelebre}
	\begin{Frase}
	\end{Frase}
	\begin{Fuente}
	\end{Fuente}
\end{FraseCelebre}

%sobre que hablamos
Automated Storytelling is a long path that began in the early 70's. The objective of this field is to provide a human quality, creativity, to an artificial intelligence system. The final purpose is to prove that the system is able to reach this capability, demonstrating that it is competent to tell a story. Though is true that this field has already been worked on and debated countless times, it is equally true that the solutions given do not cover the whole problem.
With the intention of helping the research, this project tries to create independent and unpredictable characters, ir order for them to be able of controlling autonomously and individually their own decisions and therefore create diverse stories within the same environment.

%que campo vamos a abarcar en especifico
To fulfil this purpose, a specific field called Storyteller System through Intelligent Agents exists. These agents are software entities with an advanced artificial intelligence that allows them to perceive their environment and interact with it, they are able to communicate with other agents as they try to success on the specific objectives that they will have throughout the story. These objectives will result on conflicts of interests between the different agents that will be solved through simulations, so when these simulations get finished they generate the story.

%a grandes rasgos que hemos hecho y por que
To produced this idea, a Java application is developed through a system of Intelligent Agents and planning different objectives. This application success on create a significant number of diverse stories through simulations.  
In order to be capable of doing that, is needed the platform JADE, which allows to interact with the agents, an external planner and a simulator, capable of simulate what is needed.


Following this process, which has been explored on previous investigations made in this university, is proposed a storyteller system capable of work with a larger number of agents, as well as generate characters with more possibilities and a more complex environment.


The intention with this system is that the stories take place in different worlds, with many characters capable of acting in different ways in each simulation. These characters will have their own rules and features that will defined them as unique. They will also be able of interacting with objects and will have a more diverse system to tell their own phrases, greatly favouring diversity in the storytelling.


%mini conclusion de resultados
In the following document, we will discuss the testing and solutions obtained in the development of the application and will proceed to explain the work plan. Documenting the technical basis of the technologies used and detailing in depth the work done is essential, with the objective of continuing this research and contribute with this experience for future researches related to this matter.

\section*{Keywords}
Storytelling,
Artificial Intelligence,
Multiagent System,
Simulation,
Planing.
%Generaci�n de historias,
%Sistema multiagente,
%Simulaci�n,
%Planificaci�n.

\endinput
% Variable local para emacs, para  que encuentre el fichero maestro de
% compilaci�n y funcionen mejor algunas teclas r�pidas de AucTeX
%%%
%%% Local Variables:
%%% mode: latex
%%% TeX-master: "../Tesis.tex"
%%% End:
