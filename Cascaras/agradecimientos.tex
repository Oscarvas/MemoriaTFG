%---------------------------------------------------------------------
%
%                      agradecimientos.tex
%
%---------------------------------------------------------------------
%
% agradecimientos.tex
% Copyright 2009 Marco Antonio Gomez-Martin, Pedro Pablo Gomez-Martin
%
% This file belongs to the TeXiS manual, a LaTeX template for writting
% Thesis and other documents. The complete last TeXiS package can
% be obtained from http://gaia.fdi.ucm.es/projects/texis/
%
% Although the TeXiS template itself is distributed under the 
% conditions of the LaTeX Project Public License
% (http://www.latex-project.org/lppl.txt), the manual content
% uses the CC-BY-SA license that stays that you are free:
%
%    - to share & to copy, distribute and transmit the work
%    - to remix and to adapt the work
%
% under the following conditions:
%
%    - Attribution: you must attribute the work in the manner
%      specified by the author or licensor (but not in any way that
%      suggests that they endorse you or your use of the work).
%    - Share Alike: if you alter, transform, or build upon this
%      work, you may distribute the resulting work only under the
%      same, similar or a compatible license.
%
% The complete license is available in
% http://creativecommons.org/licenses/by-sa/3.0/legalcode
%
%---------------------------------------------------------------------
%
% Contiene la p�gina de agradecimientos.
%
% Se crea como un cap�tulo sin numeraci�n.
%
%---------------------------------------------------------------------

\chapter{Agradecimientos}

\cabeceraEspecial{Agradecimientos}

\begin{FraseCelebre}
\begin{Frase}
		- Porque es lo que hacen los amigos, siempre se perdonan.
		- Ah claro si, tienes toda la raz�n. Yo te perdono ... por tu pu�alada trapera 
\end{Frase}
\begin{Fuente}
	Conversaci�n entre Asno y Shrek, Shrek
\end{Fuente}
\end{FraseCelebre}

Completar esta investigaci�n ha sido una arduo trabajo que no podr�a haberse realizado sin la ayuda de varias personas que nos han apoyado durante el desarrollo de la misma. Por esta raz�n y a todos vosotros queremos agradeceros vuestra paciencia y atenci�n. Gracias.

Gracias a nuestro director del proyecto, Gonzalo M�ndez Pozo, por aguantarnos durante tantas reuniones en las que no sac�bamos nada en claro y guiarnos por un camino que de verdad hemos podido recorrer, as� como a nuestra co-directora, Raquel Herv�s Ballesteros, que siempre nos recib�a de buen humor y nos daba �nimos y buenas ideas.


Gracias a nuestras ayudantes, Sara e Irma, por ayudarnos con buenas ideas en el dise�o del sistema y un bonito mapa.

Y por �ltimo, gracias a nuestros familiares y amigos, por apoyarnos en los momentos dif�ciles y soportar nuestros agobios, e intentar demostrarnos que si nos esforz�bamos llegar�amos a la meta propuesta. Curioso, ten�an raz�n.

\endinput
% Variable local para emacs, para  que encuentre el fichero maestro de
% compilaci�n y funcionen mejor algunas teclas r�pidas de AucTeX
%%%
%%% Local Variables:
%%% mode: latex
%%% TeX-master: "../Tesis.tex"
%%% End:
